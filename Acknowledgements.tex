\chapter*{Acknowledgements}

\begin{comment}
\begin{center}
\textit{This is dedicated to all of those with big egos}

\textit{Never fakin', we get the dough and live legal}
\end{center}
\begin{flushright}
  --Dr. Dre
\end{flushright}
\end{comment}

\begin{center}
\textit{But the theorist, before he calculated the centrifugal force and velocity of the subtile matter, should first have been certain that it existed.}
\end{center}
\begin{flushright}
  --Voltaire (c. 1778), on vortices in the aether
\end{flushright}



%\note{Need to acknowledge HPC and talk about being a stato...}

As it turns out, doing a PhD is hard. And it doesn't get any easier when the thing you're studying is as elusory as dark matter. So it goes without saying that there are plenty of people to thank for getting me to where I am now. There are far too many people whose journeys have collided with mine over the past three years to thank everyone individually, so apologies to those who don't get a name check. But my gratitude to everyone goes without saying. So much so that I'm going to say it now.

Anne Green has been a great supervisor, letting me go about my business without worrying about someone looking over my shoulder. She taught me a healthy skepticism for pretty much everything and gave me the chance to do whatever I was interested in. I also seem to have inherited her bug for running, although thankfully over much shorter distances. I should also thank Mattia Fornasa for leading by example and showing exactly how Good Science should be done. It's easy to want to cut corners when things get hard, but Mattia taught me that it's better to do things well than to do them fast. Thanks to everyone in the Nottingham Particle Theory Group, I've learned a little bit about a lot of things, as well as a hell of a lot about Inflation.

On the social side, Nottingham has been a great place to do a PhD. I've shared my office with some great people including Ken, Becky, Ippocratis, creepy Vish and Clare (who went on to bigger and better things). Thanks for making it such a great place to work. It was, of course, a pleasure to be sequestered away from the main physics building in the Astronomy-Particles enclave. Many thanks go to Lyndzo, Adam, Ian, Ewan, Jamie, Dave, Kate and Alice. A very special thanks has to go to my side-kick, Sophie, who put up with a lot as the de facto social secretary for the whole building. You all made it substantially harder to get any work done and immeasurably easier to have fun.

My family have been a wonderful support during my PhD. To name the key players: Mum, Dad, Ben, Somma, Scott and Sarah. Even if your support just involved trying to work out what it is that I've been doing for the last three years, it helped. I basically went radio-quiet for a few years and you guys managed to drag me away from the work when I needed it most. You helped me see the bigger picture and took me to pray at the altar of the only one true god: Staropramen. Calculations indicate that there should be roughly one dark matter particle per pint in the Solar neighbourhood. During the course of my PhD, I've checked plenty of pints and I've yet to find evidence of physics beyond the Standard Model.

Of course, my family has expanded in recent years. Thank you to Julie, Keith, Rachael and John-Paul (and the rest of the gang) for being welcoming, hospitable and for putting up with my feeble explanations of particle physics. Thank you in particular for helping me prop up the bar in the Kean's Head night after night after night. You dunno what it meant to me.

Finally, I thank my wife, Pip. She knew what she was letting herself in for marrying a physicist, but she did it anyway. That takes guts. For cheering me up when work was going badly and for celebrating with me 10 minutes later when I decided it was actually going well; for working just as hard as I have for the past three years; and for giving me a sense of perspective when the dark matter started closing in, I am very thankful. Pip, this work is surely dedicated to you.




