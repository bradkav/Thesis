\chapter{Speed parametrisation for directional experiments}

\section{Introduction}

While traditional direct detection experiments seek to measure the recoil energies deposited by WIMPs scattering off detector nuclei, \textit{directional} experiments aim to measure both the energy and direction of the recoil. While the recoil distribution of typical backgrounds is expected to be roughly isotropic, the WIMP-induced recoil distribution is expected to be highly directional. The motion of the Sun through the Galactic DM halo generates a so-called `WIMP wind,' leading to an event rate peaked in the opposing direction, the direction of the constellation of Cygnus. 

The ability of directional detection to distinguish background from signal and to provide a model independent confirmation of the dark matter origin of the signal make it a promising search strategy. However, measuring the direction of rare, low energy recoils remains challenging \cite{}. A number of directional detectors are currently in development and a number of novel methods for directional detection have been proposed.

Measuring the directional recoil spectrum allows us to probe not only the energy distribution of WIMPs in the Galactic halo (embodied in the speed distribution $f(v)$), but the full 3-dimensional velocity distribution $f(\textbf{v})$. This may allow us to gain new insight into the formation processes at hand in the growth of the Milky Way halo \cite{}. However, it also introduces new uncertainty into calculating the event rate. While non-directional detection leaves us with a single free function in the form of  $f(v)$, the directional case relies upon the \textit{a priori} unknown function of a 3-dimensional vector, $f(\textbf{v})$.

In this chapter, we will first introduce the formalism by which the directional rate is calculated. Specifically, we introduce the Radon transform which relates the WIMP velocity distribution to the corresponding nuclear recoil distribution.  We then discuss the current state of directional detection technology and the progress of several directional experiments. We then discuss previous approaches to mitigating the uncertainties associated with the velocity distribution. Finally, we consider a new method for parametrising $f(\textbf{v})$, which allows it to be written in terms of a finite number of one-dimensional functions, and how to calculate the Radon transform of this new, discretised distribution function.
