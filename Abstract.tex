\begin{abstract}

The detection of dark matter (DM) by direct detection experiments has great potential to shed light on particle physics beyond the Standard Model. However, uncertainties in the DM speed distribution $f_1(v)$ may lead to biased reconstructions of particle physics parameters, such as the DM mass $m_\chi$ and interaction cross sections, $\sigmapsi$ and $\sigmapsd$. In this work, we aim to determine whether these parameters can be determined from future direct detection data without any prior assumptions about $f_1(v)$.

We study previous methods for parametrising $f_1(v)$ and show that they may still lead to biased reconstructions of the DM parameters. We propose an alternative smooth, general parametrisation, which involves writing the \textit{logarithm} of the speed distribution as a polynomial in $v$. We test this method using future direct detection mock data sets and show that it allows an unbiased reconstruction of the DM mass over a range of particle physics and astrophysics parameters. However, the unknown fraction of DM particles with speeds below the energy thresholds of the experiments means that only a lower bound can be placed on the interaction cross sections.

By introducing data from neutrino telescope experiments, such as IceCube, this degeneracy in the cross section can be broken, as these experiments probe the low speed DM population. Combined with our parametrisation method, this allows a robust reconstruction of the DM mass and cross sections without relying on any assumptions about the DM speed distribution. The function $f_1(v)$ itself can also be reconstructed, allowing us to probe the distribution function of the Milky Way.

Finally, we propose a method of extending this parametrisation to directional data, which should allow even more information to be extracted from future experiments without the need for astrophysical assumptions.

\end{abstract}
