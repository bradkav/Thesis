\chapter{Conclusions}
\label{ch:Conclusions}

The presence of dark matter (DM) in the Universe has been postulated to explain a range of observations. Data from Cosmic Microwave Background anisotropies, the growth of large scale structure and the dynamics of galaxies and clusters all point towards a dark universe, with roughly 5 times as much dark matter as baryonic matter. So far, however, the detection of dark matter has only been through its gravitational influence. A number of experiments - so called direct detection experiments - are underway or in development which hope to detect weakly interacting massive particles (WIMPs) through their non-gravitational interactions in terrestrial detectors.

Once such a detection is confirmed, the next stage will be to try and measure the properties of the DM particles, such as their mass and interaction cross sections. This should help us to unravel the identity of the DM and begin to probe the structure of physics beyond the Standard Model of particle physics. However, the analysis of direct detection data is fraught with uncertainties. In this work, we have focused on astrophysical uncertainties, particularly those coming from the local speed distribution of dark matter $f_1(v)$. This distribution is \textit{a priori} unknown and a wide range of proposals having been put forward for its correct form. We cannot hope to accurately reconstruct the DM properties without first addressing these uncertainties.

Previous attempts to parametrise the DM speed distribution have been unsatisfactory. As we discuss in Chapter~\ref{ch:Speed}, these methods typically assume some specific functional form for the speed distribution, motivated by N-body simulations or assumptions of equilibrium. However, if the true shape of the speed distribution is poorly fit by the functional form assumed in the parametrisation, the particle physics parameters we are aiming to reconstruct will be biased. The aim then should be to develop a general, empirical parametrisation for the speed distribution which can be fit according to the data and which allows the DM mass and interaction cross sections to be reconstructed without bias. Such a parametrisation was first proposed by Peter \cite{Peter:2011}, in the form of a binned approximation to the speed distribution. However, this was shown to result in a bias in the reconstructed WIMP mass. 

In Chapter~\ref{ch:Speed}, we demonstrated that this bias stems from the interplay between the WIMP mass and the size of the bins in energy. For a fixed bin width in speed, varying the WIMP mass affects not only the size of the corresponding bins in energy but also the number of bins to which an experiment is sensitive. The result is that the best fit to the data may not be provided by the true underlying WIMP mass. 

This problem can be alleviated by using a binned parametrisation of the DM \textit{momentum} distribution. The range of momenta probed by a given experiment is independent of the WIMP mass, meaning that the overall shape and normalisation of the event spectrum can be probed separately. To pin down the WIMP mass, multiple experiments are required. In this case, the size and number of bins probed by each experiment depend only weakly on the WIMP mass, significantly reducing the bias seen in the binned speed parametrisation. We have also seen that the values of the momentum bin parameters may allow us to reconstruct the WIMP speed distribution itself. However, going from the reconstructed momentum distribution to the speed distribution is non-trivial. Moreover, the momentum binning method is expected to fail at low WIMP masses. Finally, this method requires a choice of the momentum range over which to parametrise, which may not always be obvious.

What properties then do we require of a general parametrisation of the WIMP distribution? It must be a physical distribution function, meaning that it must be everywhere non-negative and must be normalised. From our study of binned parametrisations, we are also lead to conclude that it should not have any fixed length scales, as these may result in a biased WIMP mass reconstruction. We have proposed that the \textit{logarithm} of directionally-averaged velocity distribution $f(v)$ should be written as a polynomial in the speed $v$ in the Earth frame. The resulting speed distribution takes the form
\begin{equation}
f_1(v) = v^2 \exp \left(-\sum_{i=0}^{N-1} a_k P_k(v)\right)\,.
\end{equation}
This ensures that $f_1(v)$ is not only strictly positive, but also a smooth function of $v$. The shape of the speed distribution is controlled by the parameters $a_k$ and the $N$ basis functions $P_k$. Logarithmic dependence on the parameters also means that a wide range of functional forms can be approximated.

In Chapter \ref{ch:poly}, we have demonstrated that using the parametrisation the WIMP mass can be reconstructed without bias over a range of benchmark masses from 10 to 500 GeV. We have also demonstrated the method for a number of possible underlying distribution functions and shown that it has the correct statistical properties when Poisson fluctuations are included. We have also set out how best to choose the basis polynomials $P_k$ and the number of basis functions.

However, direct detection experiments have finite energy thresholds, which corresponds to a minimum WIMP speed to which they are sensitive. Without information about the speed distribution below this threshold, it remains unconstrained by the experiments. This means that we do not know what fraction of WIMPs can contribute to scattering events in the detector. If we observe a small number of events at a detector, we cannot know whether they were caused by WIMPs with a low cross section or by WIMPs with a larger cross section but whose population is concentrated at low speeds. This results in a degeneracy between the cross section and the shape of the speed distribution. This degeneracy is a generic consequence of any general parametrisation of the WIMP speed distribution and means that we can only use direct detection experiments to place a lower bound on the interaction strengths of DM particles. As a consequence of this, we can only probe the \textit{shape} but not the normalisation of the WIMP speed distribution. In spite of this, it may still be possible to distinguish the Standard Halo Model from N-body speed distributions using around 1000 events.

In Chapter \ref{ch:NT}, we explore a method for breaking the speed distribution-cross section degeneracy. The capture of DM particles in the Sun is described by the same interaction cross sections which control the scattering rate in direct detection experiments. However, in the case of DM capture, it is those WIMPs with the lowest speeds which are more easily captured. Captured WIMPs then annihilate in the Sun and produce neutrinos, which can be detected at neutrino telescope experiments, such as IceCube. By combining future neutrino telescope and direct detection data, we can probe the entire range of the WIMP speed distribution. We have demonstrated this for several benchmarks. This allows us to constrain the fraction of low speed WIMPs and therefore break the degeneracy in the cross section. We have also been able to closely reconstruct the WIMP speed distribution over the entire range of interest. \note{Finish...}

\note{Need better statistical techniques to reject speed distributions...}

Finally, in Chapter \ref{ch:Directional}, we began to explore how such a parametrisation method could be extended to directional experiments. Such experiments depend on the full 3-dimensional velocity distribution $f(\textbf{v})$. Parametrising such a function is unfeasible and would require a huge number of parameters. It is therefore necessary to decompose $f(\textbf{v})$ into a smaller number of basis functions. A spherical harmonic decomposition has been suggested previously. However, the spherical harmonic basis is not strictly positive, meaning that we cannot ensure that the velocity distribution is physical at every point in parameter space. 

As an alternative, we propose an angular discretisation of the velocity distribution. As a first approximation, we have considered the forward- and backward-going distributions. However, with increasing amounts of data, it would be possible to increase the number of discretised pieces in $f(\textbf{v})$. This method allows us to constrain a small number of 1-dimensional functions $f_j(v)$, rather than a much larger space of 3-dimensional functions. We have laid out the framework for calculating the directional event rate from such a discrete parametrisation and have shown that with as few as $N=3$ discrete pieces, the event spectrum can be well fit by this approximation. 

Further work is needed to understand how this discrete approximation behaves when confronted with mock data sets. In particular, we must combine this angular discretisation with the polynomial parametrisation we have developed for the speed distribution. This will allow us to determine how many discrete pieces are required for real data sets to ensure a close enough approximate to the true recoil spectrum. We can also combine directional with non-directional data sets using the same parametrisation. The isotropic speed distribution will simply be given by the sum of each of the discrete directional pieces. Though this method remains to be tested, we have established a framework which should allow the velocity distribution to be parametrised in a tractable way.

\note{Need more `further work'...Combining with other uncertainties - particle physics...}

In this work, we have demonstrated for the first time that uncertainties in the WIMP speed distribution can be confronted and overcome in a completely general way. The polynomial $\ln f(v)$ parametrisation which we have proposed allows the WIMP mass to be reconstructed without bias, which we have demonstrated with a wide range of particle physics and astrophysics benchmarks. The introduction of neutrino telescope data allows us to probe the low speed population of WIMPs and therefore constrain not only the WIMP mass but also the WIMP interaction cross sections. We have also outlined how such a framework can be extended to incorporate directional data. This work is the first demonstration that both particle physics parameters \textit{and} the form of the speed distribution can be extracted from data from DM search experiments. \note{Need a punchy end...}

 
  

