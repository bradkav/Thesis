\chapter{Direct detection of dark matter}

\section{Introduction}

The idea that particle dark matter (DM) may be observed in terrestrial detectors was first proposed by Goodman and Witten in 1985 \cite{Goodman:1985} and by Drukier, Freese and Spergel in 1986 \cite{Drukier:1986}. If DM can interact with particles of the Standard Model, the flux of DM from the halo of the Milky Way should be large enough to cause measureable scattering from nuclei. If the subsequent recoils can be detected and their energy spectrum measured, it should be possible to infer some properties of the DM particles.

However, the expected event rate for keV-scale recoils at such a detector would be of the order of $10^{-10}$ events per kg of detector material per day per keV recoil energy \cite{Cerdeno:2010}. With such a low event rate, it is imperative that backgrounds can be reduced as much as possible. In addition, detectors should be as large as possible and sensitive to as wide a range of recoil energies as possible, in order to maximise the total number of events observed. Thus, specialised detectors are required to shield the active detector material from backgrounds and to discriminate between these backgrounds and signal events.

There exist at present a wide range of detectors using a variety of different sophisticated techniques for detecting such a weak signal against ubiquitous backgrounds, each probing a slightly different range of DM parameter space. Several of these experiments - such as DAMA/LIBRA \cite{Bernabei:2010}, CoGeNT \cite{Aalseth:2011a, Aalseth:2011b} and CRESST-II \cite{Stodolsky:2012} - claim to have observed a signal indicative of a WIMP with mass $\sim 10$ GeV. However, a number of other experiments have reported null results creating tension for a dark matter interpretation of these tentative signals. It remains to be seen whether this discrepancy is an experimental effect or physically meaningful result. 

There remain a number of uncertainties in the direct detection of dark matter. These come from a variety of sources and can be approximately partitioned into experimental, nuclear, particle and astrophysical uncertainties. Understanding these uncertainties is imperative for properly interpreting the results of direct detection experiments and understanding whether a coherent picture can emerge from a number of different experimental efforts.

In this chapter, I will review the formalism for direct detection which was introduced by Goodman, Witten, Drukier, Freese and Spergel in the 1980s (and subsequently refined). I will then briefly discuss some of the experimental techniques which are used to achieve the required sensitivity for DM searches, as well as summarising current experimental constraints and results. I will outline some of the uncertainties which afflict the interpretation of direct detection data.

I will focus on astrophysical uncertainties in direct detection. In particular, I will discuss the local density and distribution of dark matter impacts the direct detection event rate, how well we understand these different factors and review approaches which have been developed in the past to mitigate these uncertainties.

\section{Direct detection formalism}

\todo{Make sure I get the right citations and stuff in here...}
\todo{Generalise to many nuclei etc...}
\note{Make the distinction NOW about f1 or f or f3 and what I mean by that...}

\note{ELASTIC SCATTERING - what about the alternatives? Form factor DM, higher order stuff, effective operator, inelastic?}

\note{This only applies to fermionic dark matter!!!}

\note{Introduce the term WIMPs}

We wish to obtain the rate of nuclear recoils per unit detector mass. The differential event rate $R$ can be written straightforwardly as

\begin{equation}
\dbd{R}{E_R} = N_T \Phi_\chi \dbd{\sigma}{E_R}\,,
\end{equation}
for recoils of energy $E_R$, $N_T$ target particles, a DM flux of $\Phi_\chi$ and a differential scattering cross section of $\dbd{\sigma}{E_R}$. Per unit detector mass, the number of target particles is simply $N_T = 1/m_N$, for nuclei of mass $m_N$. The DM flux for particles with speed in the range $v \rightarrow v + \mathrm{d}v$ is $\Phi_\chi = n_\chi v f(v) \,\mathrm{d}v$. Here, $n_\chi$ is the number density of dark matter particles $\chi$ and $f(v)$ is the speed distribution for the dark matter. This distribution function describes the fraction of DM particles having a given speed. Finally, we can convert from the number density to the mass density $\rho_0$ by dividing by DM particle mass $m_\chi$: $n_\chi = \rho_0/m_\chi$. By integrating over all DM speeds, we therefore obtain

\begin{equation}
\dbd{R}{E_R} =  \frac{\rho_0}{m_N m_\chi} \int_{0}^{\infty} v f(v) \dbd{\sigma}{E_R} \, \mathrm{d}v\,.
\end{equation}

The differential scattering cross section per solid angle in the zero-momentum frame (ZMF), \(\Omega^*\), is given by:
\begin{equation}
\frac{\textrm{d}\sigma}{\textrm{d}\Omega^*} = \frac{1}{64 \pi^2 s} \frac{p_f^*}{p_i^*} |\mathcal{M}|^2 \,,
\end{equation}
where $\mathcal{M}$ is the scattering amplitude obtained from the Lagrangian. For elastic scattering, the final and initial momenta in the ZMF are equal: \(p_f^* = p_i^*\). The centre-of-mass energy squared, \(s\), can be written \(s \approx (m_\chi + m_N)^2\), where we have used the non-relativistic approximation \note{This is only justified later}. The recoil energy can be written in terms of the ZMF scattering angle $\theta^*$ as \cite{Cerdeno:2010}

\begin{equation}
E_R = \frac{\mu_{\chi N }^2 v^2}{m_N} (1-\cos\theta^*)\,.
\end{equation}
Noting that $\textrm{d}\Omega^* = \textrm{d}\cos\theta^*\textrm{d}\phi$, we can write:

\begin{equation}
\frac{\textrm{d}E_R}{\textrm{d}\Omega^*} = \frac{\mu_{\chi N}^2 v^2}{2\pi m_N}\,,
\end{equation}
and therefore

\begin{equation}
\dbd{\sigma}{E_R} = \frac{1}{32\pi m_N m_\chi^2 v^2}|\mathcal{M}|^2\,.
\end{equation}

The matrix element $\mathcal{M}$ is obtained from interaction terms in the lagrangian between the DM particle and quarks. This will depend on the particular DM model under consideration and the full form of these interaction terms is not known. However, because the WIMPs have speeds of order $10^{-3} c$, the scattering occurs in the non-relativistic limit, leading to some important simplifications. In this limit, the axial-vector interaction simply couples the spins of the WIMP and quark. The scalar interaction induces a coupling of the WIMP to the number of nucleons in the nucleus, with the vector\footnote{For the case of a Majorana fermion, the vector current vanishes and we need not consider it.} and tensor interactions assuming the same form as the scalar in the non-relativistic limit \cite{Jungman:1995}. All other interactions are typically suppressed by powers of $v/c$ and so will be subdominant (though we will consider briefly scenarios where this is not the case in Sec.~\ref{}). \note{Check and cite...} Generically, then, the cross section is typically written in terms of spin-independent (SI) and spin-dependent (SD) interactions \cite{Goodman:1985} \note{Talk a bit more here about effective field theories - find the right paper - there's one that has all the v/c dependences - mentioned in \cite{Engel:1992}... - only considering contact interactions, slow moving spin-1/2,...\cite{Kurylov:2003,Fan:2010,Cirelli:2013,Fitzpatrick:2013} - axial-vector and scalar currents do not interfere...}

\begin{equation}
\dbd{\sigma}{E_R} = \dbd{\sigma_{SI}}{E_R} + \dbd{\sigma_{SD}}{E_R}\,.
\end{equation}

We now discuss the form of the SI and SD cross sections in turn.

\subsection{SI interactions}

Spin-independent interactions are generated predominantly by scalar terms in the effective lagrangian\note{NB: Contact interactions in some effective field theory - what about loop diagrams...?}

\begin{equation}
\label{eq:ScalarInt}
\mathcal{L} \supset \alpha_S^{(q)} \bar{\chi} \chi \bar{q} q \,,
\end{equation}
for interactions with a quark species $q$ with coupling $\alpha_S^{(q)}$. The operator $\bar{q} q$ is simply the quark number operator, which couples to the quark density. However, we should recall that the quarks are in nucleon bound states $|n\rangle$, so we should evaluate $\langle n|\bar{q}q|n\rangle$, adding coherently the contributions from both valence and sea quarks. These matrix elements are obtained from chiral perturbation theory \cite{Alarcon:2012} or Lattice QCD \cite{Bali:2012}. These matrix elements can be parametrised in terms of their contribution to the nucleon mass in the form:

\begin{equation}
m_n f_{Tq}^n \equiv \langle n|m_q\bar{q}q|n \rangle \,.
\end{equation}

Adding the contributions of the light quarks, as well as the heavy quarks and gluons (which contribute through the chiral anomaly \cite{Shifman:1978}), we obtain

\begin{equation}
\langle n| \sum_{q,Q,g} \bar{q} q |n \rangle  = \left(\sum_{q=u,d,s}\frac{m_n}{m_q} f_{Tq}^n \alpha_S^q + \frac{2}{27} f_{TQ}^n \sum_{q = c,b,t} \frac{m_n}{m_q} \alpha_S^q\right) \equiv f^n\,.
\end{equation}
The parameters describing the contributions of the different quarks to the nucleon mass be determined experimentally. The uncertainties this produces will be discussed shortly in Sec.~\ref{DD:sec:nuclearunc}.
\note{Check - what exactly is this equal to...}

We now consider the matrix elements of the nucleon operators within a nuclear state, $|\Psi_N\rangle$:$\langle \Psi_N|f^n \bar{n}n|\Psi_N\rangle$. These operators now simply count the number of \(n\) nucleons in the nucleus, along with a momentum-dependent form factor, $F(\textbf{q})$, corresponding to the Fourier transform of the nucleon density. This takes into account the loss of coherence for nuclear scattering due to the fact that the nucleus is not point-like. We therefore obtain:
\begin{equation}
\langle \Psi_N|f^n \bar{n}n|\Psi_N\rangle = \langle \Psi_N|\Psi_N\rangle f^n N_n F_n(\textbf{q}) = 2m_N f^n N_n F_n(\textbf{q})\,,
\end{equation}
where we note that we require the wavefunctions to be normalised to \(2E \approx 2m_N\) for a nucleus of mass \(m_N\). We now add the contribution from protons to the matrix element, noting that \(F_n \approx F_p = F\) (see Sec.~\ref{DD:sec:nuclearunc})
\begin{equation}
\langle \Psi_N|f^n \bar{n}n + f^p \bar{p}p|\Psi_N\rangle = 2m_N (f^n N_n + f^p N_p) F(\textbf{q})\,,
\end{equation}
where $N_n$ and $N_p$ are the neutron and proton numbers of the nucleus respectively.

The corresponding matrix element for the scalar WIMP operator $\bar{\chi}\chi$ is simple in the non-relativistic limit, evaluating to $2 m_\chi$ \cite{}. Combining these, we obtain the scalar matrix element
\begin{equation}
|\mathcal{M}_S|^2 = 16 m_\chi^2 m_N^2 \left|f^p Z + f^n (A-Z)\right|^2 F_{SI}^2(\textbf{q})\,,
\end{equation}
and the SI cross section
\begin{equation}
\dbd{\sigma_{SI}}{E_R} = \frac{m_N}{ 2 \pi v^2} \left|f^p Z + f^n (A-Z)\right|^2 F^2(\textbf{q})\,,
\end{equation}
where we have used the atomic number $Z$ and mass number $A$ to describe the composition of the nucleus. It is conventional to write this in terms of the \note{total} WIMP-proton SI cross section, which does not depend on the particular $(A,Z)$ of the target nucleus and thus allows easy comparison between experiments. This cross section is given by

\begin{equation}
\sigma_{SI}^p = \frac{\mu_{\chi p}^2}{\pi}(f^p)^2\,,
\end{equation}
meaning that

\begin{equation}
\dbd{\sigma_{SI}}{E_R} = \frac{m_N}{ 2 \mu_{\chi p}^2 v^2} \left|Z + (f^n/f^p) (A-Z)\right|^2 F^2(E_R)\,.
\end{equation}

\todo{TALK ABOUT fn/fp.}

\todo{Talk about the vector contribution - subdominant}

\todo{Mention spin 0 and spin 1}

\note{Distinguish between nucleon and neutron with n}

\subsection{SD interactions}

The spin-dependent interaction is typically sourced by axial-vector currents of the form

\begin{equation}
\label{eq:AVInt}
\mathcal{L} \supset \alpha_{AV}^{(q)} (\bar{\chi} \gamma^\mu \gamma_5 \chi) (\bar{q} \gamma_\mu \gamma_5 q)\,.
\end{equation}
These result in a coupling of the spins of the WIMP and nucleus. In analogy with the SI case, we can write the nucleon quark matrix elements in the form \cite{Engel:1991, Engel:1992}

\begin{equation}
\langle n | \bar{q} \gamma_\mu \gamma_5 q | n \rangle = 2 s_\mu^n \Delta_q^n\,,
\end{equation}
where $s_\mu$ is the nucleon \note{/neutron} spin 4-vector and $\Delta_q^n$ parametrises the contribution of quark $q$ to this total spin. Adding the contributions of the different quarks, we can define

\begin{equation}
a_{p,n} = \sum_{q = u,d,s} \frac{\alpha_{AV}^{(q)}}{\sqrt{2}G_F} \Delta_q^{p,n}\,,
\end{equation}
which are the effective proton and neutron spin couplings. 

The full nuclear matrix elements can then be written in the form 

\begin{equation}
\langle \Psi_N | \sum_{q=u,d,s} \bar{q} \gamma_\mu \gamma_5 q | \Psi_N \rangle = 4 \sqrt{2} G_F \frac{a_p \langle S_p \rangle + a_n \langle S_N \rangle}{J} \langle \Psi_N | \hat{J} | \Psi_N \rangle F_{SD}^2(E_R)
\end{equation}
where J is the total nuclear spin, $\langle S_{p,n} \rangle$ the expectation value of the total proton and neutron spin in the nucleus and $F_{SD}^2$ is a form factor, as in the SI case, which is determined by the internal spin structure of the nucleus. \note{Should that be $4\sqrt{2}$ or $2\sqrt{2}$?} Noting that $\langle \Psi_N | \hat{J} | \Psi_N \rangle = 2J(J+1)m_N$, we obtain for the SD cross section

\begin{equation}
\dbd{\sigma_{SD}}{E_R} = \frac{16 m_N}{\pi v^2} G_F^2 \frac{J + 1}{J} \left| a_p \langle S_p \rangle + a_n \langle S_n \rangle \right|^2 F_{SD}^2(E_R)\,.
\end{equation}
\todo{Say something about the form factor - and about the `alternate' non-form factor version...Also what about the neutralino axial vector matrix element - is that just 2 mx?}

Again, as in the SI case, it is convenient to rewrite this expression in terms of the proton cross section $\sigma_{SD}^p$, which is given by \note{be more explicit about how we obtain the cross section - i.e. using the $\dbd{\sigma}{\Omega^*}$ equation...}

\begin{equation}
\sigma_{SD}^{p} = \frac{96}{4} G_F^2 \frac{\mu_{\chi p}^2}{\pi} (a_p)^2\,.
\end{equation}
This leads to the final expression for the SD cross section

\begin{equation}
\dbd{\sigma_{SD}}{E_R} = \frac{2 m_N \sigma_{SD}^p}{3 \mu_{\chi p}^2 v^2} \frac{J+1}{J} \left| \langle S_p \rangle + (a_n/a_p) \langle S_n \rangle \right|^2 F_{SD}^2(E_R)\,.
\end{equation}

\subsection{The final event rate}

It is helpful to collect these various results together to form a coherent picture of the event rate. Combining the SI and SD rates together, we can write

\begin{equation}
\dbd{\sigma}{E_R} = \frac{m_N}{2 \mu_{\chi p}^2 v^2} \left( \sigma_{SI}^p \mathcal{C}_{SI} F_{SI}^2(E_R) + \sigma_{SD}^p \mathcal{C}_{SD} F_{SD}^2(E_R) \right)\,,
\end{equation}
where the proton cross sections $\sigma_{SI,SD}^p$ were defined in the previous section, the form factors $F_{SI,SD}^2$ will be discussed in more detail in Sec.~\ref{sec:DD:nuclearunc} and we have defined the enhancement factors as 

\begin{align}
\mathcal{C}_{SI} &= \left|Z + (f^n/f^p) (A-Z)\right|^2 \\
\mathcal{C}_{SD} &= \frac{4}{3}\frac{J+1}{J} \left| \langle S_p \rangle + (a_n/a_p) \langle S_n \rangle \right|^2\,.
\end{align}

We can now incorporate these into the full event rate:

\begin{equation}
\dbd{R}{E_R} = \frac{\rho_0}{2 \mu_{\chi p}^2 m_x}\left( \sigma_{SI}^p \mathcal{C}_{SI} F_{SI}^2(E_R) + \sigma_{SD}^p \mathcal{C}_{SD} F_{SD}^2(E_R) \right) \int_{0}^\infty \frac{f(v)}{v}\,\mathrm{d}v\,.
\end{equation}

For a given experiment, which is sensitive to recoil energies in the range $E_\textrm{min}$ to $E_\textrm{max}$, the total number of events expected is obtained by integrating over this range of recoil energies and multiplying by the exposure time $t_\textrm{exp}$, detector mass $m_\textrm{det}$ and efficiency (which may also be a function of the recoil energy $E_R$) $\epsilon(E_R)$:

\begin{equation}
N_e = m_\textrm{det} t_\textrm{exp} \int_{E_\textrm{min}}^{E_\textrm{max}}\epsilon(E_R) \dbd{R}{E_R} \, \mathrm{d}E_R\,.
\end{equation}
For the case of a more realistic experiment in which the measurement of energy has only a finite resolution $\sigma(E_R)$, we convolve the event rate with a resolution function to obtain the observed recoil spectrum $\dbd{\tilde{R}}{E_R}$,
\begin{equation}
\dbd{\tilde{R}}{E_R}(E) = \int_{E' = 0}^{\infty} \frac{\mathrm{e}^{-(E - E')^2/(2\sigma(E'))}}{\sqrt{2\pi}\sigma(E')} \dbd{R}{E_R}(E') \, \mathrm{d}E'\,.
\end{equation} 

We now turn our attention to the discussion of such `realistic experiments' and the current state of dark matter direct searches. 

\note{SI good for heavier detectors...}
\note{Annual modulation}
\note{Isospin conserving assumptions...}
\todo{Add in some plots and discussion of event rates etc...}

\section{Direct detection experiments}

\note{Typical sources of backgrounds include \begin{inparaenum}[i)] \item $e/\gamma$ events, \item neutrons, \item $\alpha$-particles and \item nuclear recoils. \end{inparaenum}}

In order to measure this spectrum, a range of obstacles must be overcome. Radioactive decays due to naturally occuring isotopes may cause keV energy nuclear recoils in the detector, meaning that care must be taken to reduce their impact. The radiopurity of the target material is therefore of utmost importance (see for example Ref.~\cite{Munster:2014}), as well as the radiopurity of detector equipment itself \cite{Bernabei:2008b,Kuzniak:2012} \note{Need clean construction...}. In some cases, the naturally occurring target material is contaminated with a particular radioisotope, such as $^{39}Ar$ contamination in Argon. In these cases, special sources of the material must be found \cite{Galbiati:2008}, or the amount of contamination must be carefully measured and accounted for in data analysis \cite{Aprile:2013a} \note{I think they do actually remove the Krypton...}. \note{These are $\alpha$s and $\gamma$s I think...}

\todo{Exchange this paragraph with the one before...}
Another possible source of backgrounds are high energy cosmic rays. For this reason, direct detection experiments are typically operated underground, such as at the Gran Sasso laboratory in Italy or the Boulby laboratory in the UK, in order to reduce the penetration of these cosmic rays. However, cosmogenic neutrons can still penetrate the experiments, leading to the need for active shield which can detect these neutrons \note{and muons?} and provide a veto for any nuclear recoils they produce in the detector. Passive shielding also reduces the neutron flux from surrounding rock and other sources \note{environmental radioactivity}. For a detailed analysis of neutron sources at dark matter experiments, see Ref.~\cite{Scholl:2012} (CRESST-II) and Ref.~\cite{Aprile:2013b} (XENON100).

\note{Neutrons from U/Th contamination in the detector and surrounding materials; neutrons look exactly like WIMPs}

\note{Reject multiple events etc.; veto anti-coincident...}

A major source of backgrounds is also electron recoils, which deposit energy in the detector and must be distinguished from nuclear recoils caused by WIMP interactions. Depending on the design of the detector, different methods are used to discriminate electron from nuclear recoils and to measure the recoil energy itself. We will now summarise some of the techniques which are used.

\note{Use \href{http://cdms.phy.queensu.ca/Public_Docs/DirectDetection.html}{this list}...}
\todo{Say something about energy calibration and NR and EE...}
\todo{Which ones are sensitive to spin independent and spin dependent?}

Cryogenic experiments, such as CDMS \cite{Ahmed:2009,Ahmed:2011,Agnese:2013}, CRESST \cite{Angloher:2012}, CoGeNT \cite{Aalseth:2011a,Aalseth:2011b, Aalseth:2013,Aalseth:2014a,Aalseth:2014b} and EDELWEISS \cite{Armengaud:2011}, use cryogenic crystals of materials such as Germanium or Silicon as target materials. When a WIMP recoils from a target nucleus a phonon signal is generated in the crystal along with an ionization signal \note{be more technical - how are they measured}. By summing the energy collected in these two channels (and accounting for any which may be incompletely collected), the total energy of the nuclear recoil can be obtained. The ratio of the total nuclear recoil energy and the ionization signal is referred to as the `ionisation yield' and can be used to discriminate electron from nuclear recoils; electron recoils deposit more energy into ionisation. However, care must be taken to identify so-called `surface events' - events occurring close to the detector surface which result in an incomplete collection of ionisation signal and can thus mimic a WIMP signal.

Noble liquid experiments use liquid (or two-phase) noble elements such as Xenon and Argon as target materials. Completed or operational Xenon detectors include ZEPLIN \cite{Akimov:2012}, XENON \cite{Aprile:2011} and LUX \cite{Akerib:2014}. In these detectors, Xenon recoils produce a scintillation signal (S1) which can be observed directly using photomultiplier tubes. Ionisation electrons are also produced, which drift in an applied magnetic field, producing an electroluminescence signal (S2) in the gas phase. The sum of these signals can be used to reconstruct the total recoil energy, while the ratio S1/S2 is used to discriminate electron from nuclear recoils. \note{TPC} The two signals can also be used to localise the event within the detectors. A fiducial volume is then defined within the detector - only events inside this volume are considered in data analysis. This allows liquid Noble detectors to be self-shielding; the fiducial volume is shielded by the remaining detector volume. Experiments utilising Argon \cite{Marchionni:2011, Badertscher:2013} and Neon \cite{Boulay:2008} are currently under development, using either the scintillation to ionisation signal as a discriminant or using timing of the scintillation signal (pulse shape discrimination).

Superheated liquid detectors such as COUPP \cite{Behnke:2011}, SIMPLE \cite{Felizardo:2012} and PICASSO \cite{Archambault:2012} use a detector volume filled with droplets of superheated liquid such as $\textrm{C}_4\textrm{F}_{10}$. The deposition of kinetic energy by a WIMP will induce the nucleation of a bubble producing an acoustic signal which is detected by piezoelectric transducers. Energy deposition by other particles such as muons and $\gamma$- and $\beta$-radiation typically occurs over longer length scales and thus does not register a signal. The temperature and pressure of the detector can be tuned to specify the threshold energy, the minimum energy which must be deposited before nucleation occurs. As such, superheated liquid detectors cannot measure the energy of specific events but rather the total event rate above the energy threshold. However, by ramping up the energy threshold, the recoil spectrum can effectively be measured. \note{Sensitive to SD and light WIMPs}.

Crystal scintillator experiments \cite{Kim:2010} such as DAMA/LIBRA \cite{Bernabei:2008a,Bernabei:2010,Bernabei:2013} and KIMS \cite{Lee:2007} use crystals such as Thallium-doped Sodium Iodide (NaI(Tl)) as the detector material. When a nuclear recoil occurs with the nuclei in the crystal, scintillation occurs. The light is collected by photomultiplier tubes, with the total recoil energy being related to the amount of scintillation light produced. In the case of DAMA/LIBRA, electron-nuclear recoil discrimination is not employed. Instead, the experiment aims to observe the annual modulation of the signal which is expected due to the periodic motion of the Earth through the WIMP halo. In other cases, such as NAIAD \cite{Ahmed:2003}, pulse shape discrimination has been used to distinguish nuclear and electronic recoils.

A final class of direct detection experiments are known as `directional' direct detection experiments. These aim to measure not only the energy deposited by WIMP scattering events but also the direction of the nuclear recoils. \note{Why?} This requires the use of specialised gas time projection chambers (TPCs), which allow measurable track lengths from which the recoil direction can be determined. \note{Is any of this true?} The directional detection of dark matter is the subject of Chapter~\ref{ch:Directional} and we therefore defer a more detailed discussion until then.

\note{Neutrinos backgrounds...}

\subsection{Current limits and results}

The first major dark matter detection to be reported was that of DAMA/NaI \cite{Bernabei:2003} and its successor DAMA/LIBRA. The experiments observed an annual modulation over 13 annual cycles, with a phase matching that expected from a dark matter signal. The detection of the annual modulation has been reported at the $8.9\sigma$ confidence level over an energy range of 2-6 keV. The modulation signal was only found in single-hit events at low energies, again suggesting a dark matter origin for the signal. Such a signal is compatible with a wide range of particle physics scenarios \cite{} and has been associated with a dark matter particle of mass $m_\chi \sim 10 \textrm{ GeV}$ and SI cross section $\sigma_{SI} \sim 10^{-41} \textrm{ cm}^2$ \cite{Belli:2011}. An annual modulation signal was also observed in the CoGeNT experiment \cite{Aalseth:2011b, Aalseth:2014a}. In this case too, the period and phase are consistent with expectations, though, in both cases the amplitude of the annual modulation is approximately 5 times larger than expected. 

Excesses above the expected backgrounds have also been observed in a number of experiments.%including CoGeNT, CRESST-II and CDMS-Si. 
The CoGeNT experiment observed an exponentially rising excess of events at low energies, down to $0.5 \textrm{ keV}_{ee}$. A maximum likelihood analysis \cite{Aalseth:2014b} pointed towards a $10 \textrm{ GeV}$ WIMP interpretation, with a cross section of around $\sigma_{SI} \sim 5 \times 10^{-42} \textrm{ cm}^2$, though the significance of the `signal' lies at only 2.9$\sigma$. CRESST-II observe 67 events in the nuclear recoil signal region but expect a background of only one event due to leakage of electron recoils into this window. Taking into account other backgrounds, the CRESST-II collaboration estimate that 25-30 of these events may be due to a WIMP signal. A fit to the data produces two minima in the likelihood function: one at $m_\chi \approx 25 \textrm{ GeV}$ (in which scattering from Tungsten is appreciable) and another at $m_\chi \approx 12 \GeV$ (where Tungsten recoils lie below the energy threshold). In both cases, the fitted cross section is in the range $\sigma_{SI} \approx 10^{-42} - 5 \times 10^{-41} \textrm{ cm}^2$. Finally, a recent analysis of the Silicon detector data from CDMS-II finds 3 events in the signal region. However, the very low expected backgrounds mean that this small number of events may be significant. The probability of the known backgrounds producing these three events has been calculated at 5.4\% and a likelihood analysis shows consistency with WIMP with $m_\chi \approx 9 \GeV$ and $\sigma_{SI} \approx 2 \times 10^{-41} \cmsq$.

While it appears that a reasonably consistent picture of a low mass WIMP is emerging from several experiments \cite{Hooper:2010}, a large number of competing experiments have reported null results. Results from CDMS-II (Ge), XENON100, LUX, SuperCDMS \cite{Agnese:2014} and others set upper limits on the standard WIMP cross section several orders of magnitude lower than the claimed signal. Several explanations for this discrepancy have been offered. One possibility is background contamination of the experiments claiming to have observed a signal \cite{Kuzniak:2012}. Another possibility is that ion-channeling in the detector crystals may affect the collected ionisation signal and therefore alter the signal \cite{Bozorgnia:2010}. The DAMA/LIBRA signal has also been attributed to an annually modulated muon signal \cite{Blum:2011,Bernabei:2012} or \note{WHAT ELSE?}.

An alternative explanation is that the claimed signals \textit{are} due to a dark matter particle, but that its properties are not as simple as in the canonical case, explaining why it has not been observed in all experiments. One possibility is that the astrophysical distribution of dark matter does not match the standard assumptions. We will discuss this astrophysical distribution in more detail shortly in Sec.~\ref{DD:sec:AstroUnc}. However, it appears that even with this additional freedom, the different results cannot be reconciled \note{Fairbairn:2009,Herrero-Garcia:2012,Fox:2011b,Frandsen:2012}. A number of particle physics models have also been considered to explain the results, including spin-dependent interactions \cite{Buckley:2013}, isospin violating dark matter (for which $f^p \neq f^n$) \cite{Feng:2011}, inelastic dark matter \cite{Smith:2001} and mirror dark matter \cite{Foot:2013}. However, consistent picture which reconciles all experimental datasets remains elusive \cite{Schwetz:2011}. 

We summarise some of the completed and current direct detection experiments in Table~\ref{DD:tab:ExptSummary}. The most stringent limits on the SI WIMP-proton cross section are set by LUX \cite{Akerib:2014}, who find a limit of $\sigma_{SI}^p \leq 7.6 \times 10^{-46} \cmsq$ at a mass of $m_\chi = 33 \GeV$. The best limit for the SD cross section is set by XENON100 \cite{Aprile:2013c}: $\sigma_{SD}^p \leq 3.5 \times 10^{-40} \cmsq$. The confirmation or falsification of the signals which have been claimed thus far may have to wait for the next generation of dark matter experiments, or for corroboration from collider or indirect searches. 

\note{More of the `halo-independent' ones -Gondolo:2012,DelNobile:2013...}

\note{What about the halo-independent method?}

\begin{table}
	\begin{tabular}{ccc}
		%Give references to results and 'setup'papers...
		\hline\hline
		Experiment & Target & Status \\
		\hline
		CDMS-II (Ge) \cite{Ahmed:2009,Ahmed:2011} & Ge & Null result \\
                CDMS-II (Si) \cite{Agnese:2013} & Si & Excess \\
                SuperCDMS \cite{Agnese:2014} & Ge & Null result \\
		CoGeNT \cite{Aalseth:2011a,Aalseth:2011b, Aalseth:2013,Aalseth:2014a,Aalseth:2014b} & Ge & Excess \& annual modulation observed \\
		CRESST-II \cite{Angloher:2012} & CaWO\(_4\) & Excess observed\\
		EDELWEISS-II \cite{Armengaud:2011} &  Ge & Null result \\
		ZEPLIN-III \cite{Akimov:2012} & Xe & Null result\\
		XENON100 \cite{Aprile:2011} & Xe & Null result \\
                LUX \cite{Akerib:2014} & Xe & Null result \\
		PICASSO \cite{Archambault:2012} & \(\textrm{C}_4\textrm{F}_{10}\) & Null result \\
		SIMPLE-II \cite{Felizardo:2012} & \(\textrm{C}_2 \textrm{ClF}_5\) & Null result \\
		COUPP \cite{Behnke:2011} & CF\(_3\)I & Null result \\
		DAMA/LIBRA \cite{Bernabei:2008a,Bernabei:2010,Bernabei:2013} &  NaI(Tl) & Annual modulation observed \\
                NAIAD \cite{Ahmed:2003} & NaI(Tl) & Null result \\
		KIMS \cite{Lee:2007} & CsI(Tl) & Null result \\
		\hline\hline
		\end{tabular}
	\caption{Summary of current and completed direct detection experiments.}
	\label{DD:tab:ExptSummary}
\end{table}


\todo{Give some typical values for thresholds and efficiencies...}


\subsection{Future experiments}

Experiments which are planned or under construction typically aim to scale up the size of current detectors and reduce unwanted backgrounds (in order to increase the sensitivity to lower cross sections) as well as decrease the energy threshold (which increases sensitivity to lower masses). There are a number of ton scale detectors either in operation or planned, including XENON1T \cite{Aprile:2012}, EURECA \cite{Kraus:2007,Roth:2009} and DARWIN \cite{Baudis:2012}. With this next generation of detectors, the aim is to achieve sensitivity to the SI WIMP-proton cross section down to $\sigma_{SI}^p = 10^{-48} \cmsq$.

In addition, there have been a number of proposals for novel methods of directly detecting dark matter. These include using DNA-based detectors to provide high spatial resolution \cite{Drukier:2012}, using nano-scale explosives \cite{Lopez:2014} or charged-coupled devices \cite{Aguilar-Arevalo:2013} to achieve very low energy thresholds and using proton-beam experiments as a source for dark matter source for direct detection experiments \cite{deNiverville:2012}.

\todo{Mini-conclusion...}

\note{What about electron recoils...?}

\section{Uncertainties}

Calculation of the DM differential event rate $\dbd{R}{E_R}$ requires not only a knowledge the dark matter parameters $m_\chi$ and $\sigma_{SI,SD}$ but a number of other factors which enter into the calculation. It is important to understand how uncertainties in these different factors and parameters propagates into the event rate in order to ensure that the conclusions we draw from direct detection experiments are unbiased. These uncertainties are typically partitioned into three separate classes: nuclear physics, particle physics and astrophysics.

\subsection{Nuclear physics uncertainties}

As we have already seen, nuclear physics enters into the calculation of the nucleon matrix elements $m_n f_{Tq}^n \equiv \langle n|m_q\bar{q}q|n \rangle$. The factors $f_{Tq}^n$ must be determined either experimentally, having values

\begin{equation}
f_{Tu}^p = 0.020 \pm 0.004 ; f_{Td}^p = 0.026 \pm 0.005 ; f_{Ts}^p = 0.118 \pm 0.062\,,
\end{equation}
with $f_{Tu}^p = f_{Td}^n$, $f_{Td}^p = f_{Tu}^n$ and $f_{Ts}^p = f_{Ts}^n$. \note{Strange content is most important - justifying fp = fn...?} The main uncertainties stem from determinations of the $\pi$-nucleon sigma term, determined either experimentally from low energy pion-nucleon scattering \cite{Borasoy:1995, Pavan:2001,Alarcon:2012} or from lattice QCD calculations \cite{Bali:2012, Alvarez-Ruso:2014}. Similarly for the spin contributions $\Delta_q$ to the nucleus, values must be obtained experimentally \cite{Ashman:1988,Engel:1992},

\begin{equation}
\Delta_u = 0.78 \pm 0.08 ; \Delta_d = -0.50 \pm 0.08 ; \Delta_s = -0.16 \pm 0.08\,,
\end{equation}
although efforts are being made to obtain these values directly via calculation \cite{Qing:1998,Thomas:2008}. It should be noted that these nucleon matrix elements are only necessary if we wish to deal directly with quark-level couplings and interactions. If, instead, we consider the nucleon-level effective operators (and equivalently the WIMP-nucleon cross sections), these values are not required.

Nuclear physics also enters into the calculation of form factors, describing the internal nucleon and spin structures of the nuclei. For the SI case, the form factor is obtained from the Fourier transform of the nucleon distribution in the nucleus. The form typically used is due to Helm \cite{Helm:1956}

\begin{equation}
F_{SI}^2(E_R) = \left(\frac{3j_1(qR_1)}{qR_1}\right)^2 \mathrm{e}^{-q^2s^2}\,,
\end{equation}
where $j_1(x)$ is a spherical bessel function of the first kind,
\begin{equation}
j_1(x) = \frac{\sin x}{x^2} - \frac{\cos x}{x}\,.
\end{equation}
Typically used are nuclear parameters due to Lewin and Smith \cite{Lewin:1996}, based on fits to muon spectroscopy data \cite{Fricke:1995}:
\begin{align}
R_1 & = \sqrt{c^2 + \frac{7}{3}\pi^2a^2 - 5s^2} \\
c & = 1.23A^{1/3} - 0.60 \mathrm{ fm} \\
a & = 0.52 \mathrm{ fm} \\
s & = 0.9 \mathrm{ fm} \,.
\end{align}
Muon spectroscopy is used as a probe of the \textit{charge} distribution in the nucleus. However, detailed Hartree-Fock calculations indicate that the charge distribution can be used as a good proxy for the nucleon distribution (especially in the case $f_p \approx f_n$) and that using the approximate Helm form factor introduces an error of less than $\sim$5\% in the total event rate \cite{Co:2012}. Studies also indicate that errors due to distortions in nuclear shape away from sphericity are negligible \cite{Ya-Zheng:2012}.

\note{Find that paper that compares all the different SI form factors...}

In the SD case, however, the situation is more complicated...\note{Copy over from Complementarity.tex}

\subsection{Particle physics uncertainties}

What do I really want to say here? Different types of interactions - just a brief summary.

\subsection{Astrophysical uncertainties}

\note{COPY TONS FROM OLD PAPERS - ESP. THE ANNIKA REVIEW PAPER}

