\chapter{Direct detection of dark matter}

\section{Introduction}

The idea that particle dark matter (DM) may be observed in terrestrial detectors was first proposed by Goodman and Witten in 1985 \cite{Goodman:1985} and by Drukier, Freese and Spergel in 1986 \cite{Drukier:1986}. If DM can interact with particles of the Standard Model, the flux of DM from the halo of the Milky Way should be large enough to cause measureable scattering from nuclei. If the subsequent recoils can be detected and their energy spectrum measured, it should be possible to infer some properties of the DM particles.

However, the expected event rate for keV-scale recoils at such a detector would be of the order of $10^{-10}$ events per kg of detector material per day per keV recoil energy \cite{Cerdeno:2010}. With such a low event rate, it is imperative that backgrounds can be reduced as much as possible. In addition, detectors should be as large as possible and sensitive to as wide a range of recoil energies as possible, in order to maximise the total number of events observed. Thus, specialised detectors are required to shield the active detector material from backgrounds and to discriminate between these backgrounds and signal events.

There exist at present a wide range of detectors using a variety of different sophisticated techniques for detecting such a weak signal against ubiquitous backgrounds, each probing a slightly different range of DM parameter space. Several of these experiments - such as DAMA/LIBRA \cite{Bernabei:2010}, CoGeNT \cite{Aalseth:2011a, Aalseth:2011b} and CRESST-II \cite{Stodolsky:2012} - claim to have observed a signal indicative of a WIMP with mass $\sim 10$ GeV. However, a number of other experiments have reported null results creating tension for a dark matter interpretation of these tentative signals. It remains to be seen whether this discrepancy is an experimental effect or physically meaningful result. 

There remain a number of uncertainties in the direct detection of dark matter. These come from a variety of sources and can be approximately partitioned into experimental, nuclear, particle and astrophysical uncertainties. Understanding these uncertainties is imperative for properly interpreting the results of direct detection experiments and understanding whether a coherent picture can emerge from a number of different experimental efforts.

In this chapter, I will review the formalism for direct detection which was introduced by Goodman, Witten, Drukier, Freese and Spergel in the 1980s (and subsequently refined). I will then briefly discuss some of the experimental techniques which are used to achieve the required sensitivity for DM searches, as well as summarising current experimental constraints and results. I will outline some of the uncertainties which afflict the interpretation of direct detection data.

I will focus on astrophysical uncertainties in direct detection. In particular, I will discuss the local density and distribution of dark matter impacts the direct detection event rate, how well we understand these different factors and review approaches which have been developed in the past to mitigate these uncertainties.

\section{Direct detection formalism}

\todo{Make sure I get the right citations and stuff in here...}
\todo{Generalise to many nuclei etc...}
\note{Make the distinction NOW about f1 or f or f3 and what I mean by that...}

\note{ELASTIC SCATTERING - what about the alternatives? Form factor DM, higher order stuff, effective operator, inelastic?}

\note{This only applies to fermionic dark matter!!!}

\note{Introduce the term WIMPs}

We wish to obtain the rate of nuclear recoils per unit detector mass. The differential event rate $R$ can be written straightforwardly as

\begin{equation}
\dbd{R}{E_R} = N_T \Phi_\chi \dbd{\sigma}{E_R}\,,
\end{equation}
for recoils of energy $E_R$, $N_T$ target particles, a DM flux of $\Phi_\chi$ and a differential scattering cross section of $\dbd{\sigma}{E_R}$. Per unit detector mass, the number of target particles is simply $N_T = 1/m_N$, for nuclei of mass $m_N$. The DM flux for particles with speed in the range $v \rightarrow v + \mathrm{d}v$ is $\Phi_\chi = n_\chi v f(v) \,\mathrm{d}v$. Here, $n_\chi$ is the number density of dark matter particles $\chi$ and $f(v)$ is the speed distribution for the dark matter. This distribution function describes the fraction of DM particles having a given speed. Finally, we can convert from the number density to the mass density $\rho_0$ by dividing by DM particle mass $m_\chi$: $n_\chi = \rho_0/m_\chi$. By integrating over all DM speeds, we therefore obtain

\begin{equation}
\dbd{R}{E_R} =  \frac{\rho_0}{m_N m_\chi} \int_{0}^{\infty} v f(v) \dbd{\sigma}{E_R} \, \mathrm{d}v\,.
\end{equation}

The differential scattering cross section per solid angle in the zero-momentum frame (ZMF), \(\Omega^*\), is given by:
\begin{equation}
\frac{\textrm{d}\sigma}{\textrm{d}\Omega^*} = \frac{1}{64 \pi^2 s} \frac{p_f^*}{p_i^*} |\mathcal{M}|^2 \,,
\end{equation}
where $\mathcal{M}$ is the scattering amplitude obtained from the Lagrangian. For elastic scattering, the final and initial momenta in the ZMF are equal: \(p_f^* = p_i^*\). The centre-of-mass energy squared, \(s\), can be written \(s \approx (m_\chi + m_N)^2\), where we have used the non-relativistic approximation \note{This is only justified later}. The recoil energy can be written in terms of the ZMF scattering angle $\theta^*$ as \cite{Cerdeno:2010}

\begin{equation}
E_R = \frac{\mu_{\chi N }^2 v^2}{m_N} (1-\cos\theta^*)\,.
\end{equation}
Noting that $\textrm{d}\Omega^* = \textrm{d}\cos\theta^*\textrm{d}\phi$, we can write:

\begin{equation}
\frac{\textrm{d}E_R}{\textrm{d}\Omega^*} = \frac{\mu_{\chi N}^2 v^2}{2\pi m_N}\,,
\end{equation}
and therefore

\begin{equation}
\dbd{\sigma}{E_R} = \frac{1}{32\pi m_N m_\chi^2 v^2}|\mathcal{M}|^2\,.
\end{equation}

The matrix element $\mathcal{M}$ is obtained from interaction terms in the lagrangian between the DM particle and quarks. This will depend on the particular DM model under consideration and the full form of these interaction terms is not known. However, because the WIMPs have speeds of order $10^{-3} c$, the scattering occurs in the non-relativistic limit, leading to some important simplifications. In this limit, the axial-vector interaction simply couples the spins of the WIMP and quark. The scalar interaction induces a coupling of the WIMP to the number of nucleons in the nucleus, with the vector\footnote{For the case of a Majorana fermion, the vector current vanishes and we need not consider it.} and tensor interactions assuming the same form as the scalar in the non-relativistic limit \cite{Jungman:1995}. All other interactions are typically suppressed by powers of $v/c$ and so will be subdominant (though we will consider briefly scenarios where this is not the case in Sec.~\ref{}). \note{Check and cite...} Generically, then, the cross section is typically written in terms of spin-independent (SI) and spin-dependent (SD) interactions \cite{Goodman:1985} \note{Talk a bit more here about effective field theories - find the right paper - there's one that has all the v/c dependences - mentioned in \cite{Engel:1992}... - only considering contact interactions, slow moving spin-1/2,...\cite{Kurylov:2003,Fan:2010,Cirelli:2013,Fitzpatrick:2013} - axial-vector and scalar currents do not interfere...}

\begin{equation}
\dbd{\sigma}{E_R} = \dbd{\sigma_{SI}}{E_R} + \dbd{\sigma_{SD}}{E_R}\,.
\end{equation}

We now discuss the form of the SI and SD cross sections in turn.

\subsection{SI interactions}

Spin-independent interactions are generated predominantly by scalar terms in the effective lagrangian\note{NB: Contact interactions in some effective field theory - what about loop diagrams...?}

\begin{equation}
\label{eq:ScalarInt}
\mathcal{L} \supset \alpha_S^{(q)} \bar{\chi} \chi \bar{q} q \,,
\end{equation}
for interactions with a quark species $q$ with coupling $\alpha_S^{(q)}$. The operator $\bar{q} q$ is simply the quark number operator, which couples to the quark density. However, we should recall that the quarks are in nucleon bound states $|n\rangle$, so we should evaluate $\langle n|\bar{q}q|n\rangle$, adding coherently the contributions from both valence and sea quarks. These matrix elements are obtained from chiral perturbation theory \cite{Alarcon:2012} or Lattice QCD \cite{Bali:2012}. These matrix elements can be parametrised in terms of their contribution to the nucleon mass in the form:

\begin{equation}
m_n f_{Tq}^n \equiv \langle n|m_q\bar{q}q|n \rangle \,.
\end{equation}

Adding the contributions of the light quarks, as well as the heavy quarks and gluons (which contribute through the chiral anomaly \cite{Shifman:1978}), we obtain

\begin{equation}
\langle n| \sum_{q,Q,g} \bar{q} q |n \rangle  = \left(\sum_{q=u,d,s}\frac{m_n}{m_q} f_{Tq}^n \alpha_S^q + \frac{2}{27} f_{TQ}^n \sum_{q = c,b,t} \frac{m_n}{m_q} \alpha_S^q\right) \equiv f^n\,.
\end{equation}
The parameters describing the contributions of the different quarks to the nucleon mass be determined experimentally. The uncertainties this produces will be discussed shortly in Sec.~\ref{DD:sec:nuclearunc}.
\note{Check - what exactly is this equal to...}

We now consider the matrix elements of the nucleon operators within a nuclear state, $|\Psi_N\rangle$:$\langle \Psi_N|f^n \bar{n}n|\Psi_N\rangle$. These operators now simply count the number of \(n\) nucleons in the nucleus, along with a momentum-dependent form factor, $F(\textbf{q})$, corresponding to the Fourier transform of the nucleon density. This takes into account the loss of coherence for nuclear scattering due to the fact that the nucleus is not point-like. We therefore obtain:
\begin{equation}
\langle \Psi_N|f^n \bar{n}n|\Psi_N\rangle = \langle \Psi_N|\Psi_N\rangle f^n N_n F_n(\textbf{q}) = 2m_N f^n N_n F_n(\textbf{q})\,,
\end{equation}
where we note that we require the wavefunctions to be normalised to \(2E \approx 2m_N\) for a nucleus of mass \(m_N\). We now add the contribution from protons to the matrix element, noting that \(F_n \approx F_p = F\) (see Sec.~\ref{DD:sec:nuclearunc})
\begin{equation}
\langle \Psi_N|f^n \bar{n}n + f^p \bar{p}p|\Psi_N\rangle = 2m_N (f^n N_n + f^p N_p) F(\textbf{q})\,,
\end{equation}
where $N_n$ and $N_p$ are the neutron and proton numbers of the nucleus respectively.

The corresponding matrix element for the scalar WIMP operator $\bar{\chi}\chi$ is simple in the non-relativistic limit, evaluating to $2 m_\chi$ \cite{}. Combining these, we obtain the scalar matrix element
\begin{equation}
|\mathcal{M}_S|^2 = 16 m_\chi^2 m_N^2 \left|f^p Z + f^n (A-Z)\right|^2 F_{SI}^2(\textbf{q})\,,
\end{equation}
and the SI cross section
\begin{equation}
\dbd{\sigma_{SI}}{E_R} = \frac{m_N}{ 2 \pi v^2} \left|f^p Z + f^n (A-Z)\right|^2 F^2(\textbf{q})\,,
\end{equation}
where we have used the atomic number $Z$ and mass number $A$ to describe the composition of the nucleus. It is conventional to write this in terms of the \note{total} WIMP-proton SI cross section, which does not depend on the particular $(A,Z)$ of the target nucleus and thus allows easy comparison between experiments. This cross section is given by

\begin{equation}
\sigma_{SI}^p = \frac{\mu_{\chi p}^2}{\pi}(f^p)^2\,,
\end{equation}
meaning that

\begin{equation}
\dbd{\sigma_{SI}}{E_R} = \frac{m_N}{ 2 \mu_{\chi p}^2 v^2} \left|Z + (f^n/f^p) (A-Z)\right|^2 F^2(E_R)\,.
\end{equation}

\todo{TALK ABOUT fn/fp.}

\todo{Talk about the vector contribution - subdominant}

\todo{Mention spin 0 and spin 1}

\note{Distinguish between nucleon and neutron with n}

\subsection{SD interactions}

The spin-dependent interaction is typically sourced by axial-vector currents of the form

\begin{equation}
\label{eq:AVInt}
\mathcal{L} \supset \alpha_{AV}^{(q)} (\bar{\chi} \gamma^\mu \gamma_5 \chi) (\bar{q} \gamma_\mu \gamma_5 q)\,.
\end{equation}
These result in a coupling of the spins of the WIMP and nucleus. In analogy with the SI case, we can write the nucleon quark matrix elements in the form \cite{Engel:1991, Engel:1992}

\begin{equation}
\langle n | \bar{q} \gamma_\mu \gamma_5 q | n \rangle = 2 s_\mu^n \Delta_q^n\,,
\end{equation}
where $s_\mu$ is the nucleon \note{/neutron} spin 4-vector and $\Delta_q^n$ parametrises the contribution of quark $q$ to this total spin. Adding the contributions of the different quarks, we can define

\begin{equation}
a_{p,n} = \sum_{q = u,d,s} \frac{\alpha_{AV}^{(q)}}{\sqrt{2}G_F} \Delta_q^{p,n}\,,
\end{equation}
which are the effective proton and neutron spin couplings. 

The full nuclear matrix elements can then be written in the form 

\begin{equation}
\langle \Psi_N | \sum_{q=u,d,s} \bar{q} \gamma_\mu \gamma_5 q | \Psi_N \rangle = 4 \sqrt{2} G_F \frac{a_p \langle S_p \rangle + a_n \langle S_N \rangle}{J} \langle \Psi_N | \hat{J} | \Psi_N \rangle F_{SD}^2(E_R)
\end{equation}
where J is the total nuclear spin, $\langle S_{p,n} \rangle$ the expectation value of the total proton and neutron spin in the nucleus and $F_{SD}^2$ is a form factor, as in the SI case, which is determined by the internal spin structure of the nucleus. \note{Should that be $4\sqrt{2}$ or $2\sqrt{2}$?} Noting that $\langle \Psi_N | \hat{J} | \Psi_N \rangle = 2J(J+1)m_N$, we obtain for the SD cross section

\begin{equation}
\dbd{\sigma_{SD}}{E_R} = \frac{16 m_N}{\pi v^2} G_F^2 \frac{J + 1}{J} \left| a_p \langle S_p \rangle + a_n \langle S_n \rangle \right|^2 F_{SD}^2(E_R)\,.
\end{equation}
\todo{Say something about the form factor - and about the `alternate' non-form factor version...Also what about the neutralino axial vector matrix element - is that just 2 mx?}

Again, as in the SI case, it is convenient to rewrite this expression in terms of the proton cross section $\sigma_{SD}^p$, which is given by \note{be more explicit about how we obtain the cross section - i.e. using the $\dbd{\sigma}{\Omega^*}$ equation...}

\begin{equation}
\sigma_{SD}^{p} = \frac{96}{4} G_F^2 \frac{\mu_{\chi p}^2}{\pi} (a_p)^2\,.
\end{equation}
This leads to the final expression for the SD cross section

\begin{equation}
\dbd{\sigma_{SD}}{E_R} = \frac{2 m_N \sigma_{SD}^p}{3 \mu_{\chi p}^2 v^2} \frac{J+1}{J} \left| \langle S_p \rangle + (a_n/a_p) \langle S_n \rangle \right|^2 F_{SD}^2(E_R)\,.
\end{equation}

\subsection{The final event rate}

It is helpful to collect these various results together to form a coherent picture of the event rate. Combining the SI and SD rates together, we can write

\begin{equation}
\dbd{\sigma}{E_R} = \frac{m_N}{2 \mu_{\chi p}^2 v^2} \left( \sigma_{SI}^p \mathcal{C}_{SI} F_{SI}^2(E_R) + \sigma_{SD}^p \mathcal{C}_{SD} F_{SD}^2(E_R) \right)\,,
\end{equation}
where the proton cross sections $\sigma_{SI,SD}^p$ were defined in the previous section, the form factors $F_{SI,SD}^2$ will be discussed in more detail in Sec.~\ref{sec:DD:nuclearunc} and we have defined the enhancement factors as 

\begin{align}
\mathcal{C}_{SI} &= \left|Z + (f^n/f^p) (A-Z)\right|^2 \\
\mathcal{C}_{SD} &= \frac{4}{3}\frac{J+1}{J} \left| \langle S_p \rangle + (a_n/a_p) \langle S_n \rangle \right|^2\,.
\end{align}

We can now incorporate these into the full event rate:

\begin{equation}
\dbd{R}{E_R} = \frac{\rho_0}{2 \mu_{\chi p}^2 m_x}\left( \sigma_{SI}^p \mathcal{C}_{SI} F_{SI}^2(E_R) + \sigma_{SD}^p \mathcal{C}_{SD} F_{SD}^2(E_R) \right) \int_{0}^\infty \frac{f(v)}{v}\,\mathrm{d}v\,.
\end{equation}

For a given experiment, which is sensitive to recoil energies in the range $E_\textrm{min}$ to $E_\textrm{max}$, the total number of events expected is obtained by integrating over this range of recoil energies and multiplying by the exposure time $t_\textrm{exp}$, detector mass $m_\textrm{det}$ and efficiency (which may also be a function of the recoil energy $E_R$) $\epsilon(E_R)$:

\begin{equation}
N_e = m_\textrm{det} t_\textrm{exp} \int_{E_\textrm{min}}^{E_\textrm{max}}\epsilon(E_R) \dbd{R}{E_R} \, \mathrm{d}E_R\,.
\end{equation}
For the case of a more realistic experiment in which the measurement of energy has only a finite resolution $\sigma(E_R)$, we convolve the event rate with a resolution function to obtain the observed recoil spectrum $\dbd{\tilde{R}}{E_R}$,
\begin{equation}
\dbd{\tilde{R}}{E_R}(E) = \int_{E' = 0}^{\infty} \frac{\mathrm{e}^{-(E - E')^2/(2\sigma(E'))}}{\sqrt{2\pi}\sigma(E')} \dbd{R}{E_R}(E') \, \mathrm{d}E'\,.
\end{equation} 

We now turn our attention to the discussion of such `realistic experiments' and the current state of dark matter direct searches. 

\note{SI good for heavier detectors...}
\note{Annual modulation}

\section{Direct detection experiments}


\begin{table}
	\begin{tabular}{ccc}
		%Give references to results and 'setup'papers...
		\hline\hline
		Experiment & Target & Status \\
		\hline
		CDMS-II \cite{Ahmed:2009,Ahmed:2011,Agnese:2013} & Ge, Si & Null result \\
		CoGeNT \cite{Aalseth:2011a,Aalseth:2011b} & Ge & Excess \& annual modulation observed \\
		XENON100 \cite{Aprile:2011} & Xe & Null result \\
		CRESST-II \cite{Angloher:2012} & CaWO\(_4\) & Excess observed\\
		DAMA/LIBRA \cite{Bernabei:2010} &  NaI(Tl) & Annual modulation observed \\
		EDELWEISS-II \cite{Armengaud:2011} &  Ge & Null result \\
		ZEPLIN-III \cite{Akimov:2012} & Xe & Null result\\
		KIMS \cite{Lee:2007} & CsI(Tl) & Null result \\
		PICASSO \cite{Archambault:2012} & \(^{19}\textrm{F}\) & Null result \\
		SIMPLE-II \cite{Felizardo:2012} & \(\textrm{C}_2 \textrm{ClF}_5\) & Null result \\
		COUPP \cite{Behnke:2011} & CF\(_3\)I & Null result \\
		\hline\hline
		\end{tabular}
	\caption{Summary of current WIMP detection experiments.}
	\label{DD:tab:ExptSummary}
\end{table}



\subsection{Directional experiments}







\todo{Add in some plots of event rates etc...}
PION-NUCLEON sigma term \cite{Borasoy:1995, Pavan:2001,Alarcon:2012}




